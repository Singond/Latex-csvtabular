% A testing document for the csvtabular package.
\documentclass{article}
\usepackage{booktabs}
\usepackage{csvsimple}
\usepackage{multicol}
\newcommand{\coltext}[1]{\multicolumn{1}{c}{#1}} % For column headers
\usepackage{forloop}
\usepackage{pgffor}

\usepackage{csvtabular}

\begin{document}

\section{Experiment}

\newcommand{\tablekclheader}{
	\coltext{$\theta$} &
	\coltext{$R$} &
	\coltext{$C$}
	\\
	\coltext{$[\mathrm{C}]$} &
	\coltext{$[\mathrm{\Omega}]$} &
	\coltext{$[\mathrm{pF}]$}
}
\begin{table}[htp]
	\small
	\caption{Resistance and capacity of cell with KCl solution}
	\label{tab:kcl-one}
	\centering
	\csvtabular[breaks=30, spacer=\quad, from=8]
		{measurement-capacity.csv}
		{c c c}
		{\tablekclheader}
		{\csvcoli & \csvcolii & \csvcoliii}
		{}
\end{table}
% Note that first line is always ignored, even if it contains numeric data
\begin{table}[htp]
	\small
	\caption{Resistance and capacity of cell with KCl solution,
		using a CSV table without a header line}
	\label{tab:kcl-two}
	\centering
	\csvtabular[breaks=26, spacer=\quad]
		{measurement-capacity-nohead.csv}
		{c c c}
		{\tablekclheader}
		{\csvcoli & \csvcolii & \csvcoliii}
		{}
\end{table}
\begin{table}[htp]
	\small
	\caption{Resistance and capacity of cell with KCl solution}
	\label{tab:kcl-three}
	\centering
	\csvtabular[breaks={17,37}, position=t]
		{measurement-capacity.csv}
		{c c c}
		{\tablekclheader}
		{\csvcoli & \csvcolii & \csvcoliii}
		{\coltext{$\hat\mu$} & 650.0 & 0.2294}
\end{table}

% Two adjacent csvtabulars
\begin{table}[htp]
	\small
	\caption{Space between adjacent \tt{csvtabulars}}
	\label{tab:kcl-four}
	\centering
	\csvtabular[from=15, position=t]
		{measurement-capacity.csv}
		{c c c}
		{\tablekclheader}
		{\csvcoli & \csvcolii & \csvcoliii}
		{\coltext{$\hat\mu$} & 650.0 & 0.2294}%
	\csvtabular[from=15, position=t]
		{measurement-capacity.csv}
		{c c c}
		{\tablekclheader}
		{\csvcoli & \csvcolii & \csvcoliii}
		{\coltext{$\hat\mu$} & 650.0 & 0.2294}
\end{table}

\end{document}
